\documentclass[a4paper]{exam}


\usepackage{amsfonts,amsmath,amsthm}
\usepackage[a4paper]{geometry}
\usepackage{hyperref}
\usepackage{xcolor}

\theoremstyle{definition}
\newtheorem{definition}{Definition}

\newcommand\Z{\ensuremath{\mathbb{Z}}}
\newcommand\R{\ensuremath{\mathbb{R}}}

\title{Problem Set 08: Sets And Functions}
\author{CS/MATH 113 Discrete Mathematics}
\date{Spring 2024}

\boxedpoints

\printanswers

\begin{document}
\maketitle

\begin{questions}
\question 
  Prove: $\mathcal{P}(A) \subseteq \mathcal{P}(B)$ if and only if $A \subseteq B$.

  \begin{solution}
    % Enter your solution here.
    \begin{proof}
      We need to prove that \\
      1. $A \subseteq B \implies \mathcal{P}(A) \subseteq \mathcal{P}(B)$\\
      2. $\mathcal{P}(A) \subseteq \mathcal{P}(B) \implies A \subseteq B$\\\\
      considering (1):\\
      Assume $A \subseteq B$\\
      then \(mathcal{P}(A) = \{C | C \subseteq A\}\)\\
      Since every C in \(mathcal{P}(A)\) belongs to B\\
      Therefore, \(\mathcal{P}(A) \subseteq \mathcal{P}(B)\)\\\\
      considering (2):\\
      Assume $\mathcal{P}(A) \subseteq \mathcal{P}(B)$\\
      Since \(A \in \mathcal{P}(A) and A \in \mathcal{P}(B)\)\\
      Therefore, \(A \subseteq B\)

      
    \end{proof}
  \end{solution}


\question Show that if \(A \subseteq C\) and \(B \subseteq D\), then \(A \times B \subseteq C \times D\).

  \begin{solution}
    \begin{proof}
    Considering the premise, \(A \subseteq C\) and \(B \subseteq D\)\\
    \(A \times B = \{(a,b) | a \in A, b \in B\}\)\\
    Since \(A \subseteq C\) and \(a \in A\), it should also be a member of \(C\)\\
    Similarly, we can say that \(b \in B\) should also be a member of \(D\)\\
    Therefore, every element \((a,b)\) in \(A \times B\), also belongs to \(C \times D\)
    \end{proof}
  \end{solution}
  
\question Prove or disprove the following statements for all sets $A, B,$ and $C$:
  \begin{parts}
  \part $ A \times (B \cup C) = (A \times B) \cup (A \times C) $
    \begin{solution}
      % Enter your solution here.
      \begin{proof}
        
        \(B \cup C = \{x | x \in B \lor x \in C\}\)\\
        \(A \times (B \cup C)\ = \{(a, x) | a \in A, x \in B \lor x \in C\}\)\\
        Since $(A \times B) \cup (A \times C)$ will also contain ordered pairs \((a, x)\)\\
        such that the $x \in B \lor x \in C$\\
        Therefore \((A \times B) \cup (A \times C)\ = \{(a, x) | a \in A, x \in B \lor x \in C\}\)
        

        
      \end{proof}
    \end{solution}

  \part $A \times (B \cap C) = (A \times B) \cap (A \times C) $
    \begin{solution}
      % Enter your solution here.
      \begin{proof}
        \(B \cap C = \{x | x \in B \land x \in C\}\)\\
        Then \(A \times (B \cap C) = \{(a, x) | a \in A, x \in B \land x \in C\}\)\\
        Thus according to the definintion of intersecton \((a, x) \in A \times B\) and \((a, x) \in A \times C\) \\
        Therefore \((A \times B) \cap (A \times C)\)\\

        Similarly, if there is an ordered pair \((a, b)\), such that \(a \in A\) and \(b \in B\)\\
        then \(b \in C\), thus \(b \in (B \cap C)\)
        Therefore \(A \times (B \cap C)\) 
      \end{proof}
    \end{solution}
  \end{parts}
  
\question If $A, B,$ and $C$ are sets such that $A \subseteq B$ and $B \subseteq C$, show that $A \subseteq C$.

  \begin{solution}
    % Enter your solution here.
    \begin{proof}
     Given that $\forall x (x \in A \implies x \in B)$ and $\forall x (x \in A \implies x \in B)$
     Through universal instantiation, :\\
  
      1. $t \in A \implies t \in B$\\
      2. $t \in B \implies t \in C$\\
      where t is an arbitary element\\\\

      Then through disjuctive syllogism: $t \in A \implies t \in C$\\
      Therefore, universal generalisation results in  $\forall x (x \in A \implies x \in C)$
  
    \end{proof}
  \end{solution}

\question Show that if $A$ and $B$ are finite sets, then  \( A \cup B \) is a finite set.

  \begin{solution}
    % Enter your solution here.
    \begin{proof}
    for \(A \cup B\) to be a finite set \(|A \cup B|\) should be a non-negative finite number\\
    We know that \(|A \cup B| = |A| + |B| - |A \cap B|\)\\
    Since \(A\) and \(B\) are finite then \(|A|\) and \(|B|\) are also non-negative finite numbers\\
    We also know that \(|A \cap B| \leq |A| \land |A \cap B| \leq |B|\)\\
    Hence, performing the above arithmetic operations results in a non-negative finite number.  
    \end{proof}
  \end{solution}
  
\question
  \begin{parts}
  \part Prove that the function $f: \Z\to\Z$, defined as \(f(n) = n^2\) is neither injective nor surjective.
    \begin{solution}
      % Enter your solution here.\
      \begin{proof}
        Since injection means $\forall x \forall y (f(x) = f(y) \implies x = y)$\\
        Consider x = 2 and y = -2 then $f(2) = 4$ and $f(-2) = 4$\\
        since $f(2) = f(-2) \implies 2 \neq -2$, the implication is false\\
        Hence the function is not injective 
        \\
        Since surjection means $\forall y \exists x (f(x) = y)$\\
        Considering that the codomain consists of all the integers thus there should be a value of x which maps to all integers\\
        But, since the range only consists of positive integer the function is not surjective\\
        Consider f(x) = -1, there exists no x which maps onto such value. 
      \end{proof}
    \end{solution}

  \part Can you make this function injective without doing any changes to the original function? (Hint: try redefining its domain and/or co-domain.)
    \begin{solution}
      % Enter your solution here.
      by defining the domain as  $f: \Z^+\to\Z^+$
    \end{solution}
  \end{parts}

  
\question Prove that the function $f: \Z\to\Z$ defined as \(f(n) = n+1\), is invertible.

  \begin{solution}
    % Enter your solution here.
    \begin{proof}
      For f(n) to invertible, it should be bijective thus we need to prove injection and surjection.\\\\
      Surjection can be proved using proof by contradiction\\
      consider $f(x) = f(y)$ and $x \neq y$\\
      then,
      \[
      x + 1 = y + 1
      x = y  
      \] 
      Hence there is contradiction and injection is proved\\\\
      To prove surjection\\
      consider $y \in \Z$\\
      then $f(y-1) = y$
    \end{proof}
  \end{solution}
  
\question Give an explicit formula for a function from \Z to $\Z^+$ that is:
  \begin{parts}
  \part one-to-one, but not onto,
    \begin{solution}
      % Enter your solution here.
      $f(x) = 2x + 3, x \geq 0$
    \end{solution}

  \part onto, but not one-to-one,
    \begin{solution}
      % Enter your solution here.
      $f(x) = |x| + 1$
    \end{solution}

  \part one-to-one and onto, and
    \begin{solution}
      % Enter your solution here.
      $f(x) = 2x + 1, x \geq 0$
      $f(x) = -2x, x < 0$
    \end{solution}

  \part neither one-to-one nor onto.
    \begin{solution}
      % Enter your solution here.
      $f(x) = x^2 + 1$
    \end{solution}
  \end{parts}

\question A function \(f: I \rightarrow \R\) is strictly increasing on an interval \(I\) if for all \(x_1, x_2 \in I\) with \(x_1 < x_2\), it holds that \(f(x_1) < f(x_2)\). Here $I$ represents the interval on which the function is strictly increasing.
  
  Given this definition, prove that a strictly increasing function is always injective over the interval, I.

  \begin{solution}
    % Enter your solution here.
    \begin{proof}
    The definition of injection is $\forall x \forall y (f(x) = f(y) \implies x = y)$\\
    since $f(x1) < f(x2)$ over the interval I\\
    thus $f(x1) \neq f(x2)$\\
    Thus injection is vacuosly proved as the premise of $f(x) = f(y)$ is always false 
    \end{proof}
  \end{solution}
\end{questions}
\end{document}
%%% Local Variables:
%%% mode: latex
%%% TeX-master: t
%%% End: