\documentclass[a4paper]{exam}

\usepackage{amsfonts,amsmath,amsthm}
\usepackage[a4paper]{geometry}
\usepackage{graphicx}
\usepackage{hyperref}
\graphicspath{{images/}}

\header{CS/MATH 113}{WC11: Mathematical Induction}{Spring 2024}
\footer{}{Page \thepage\ of \numpages}{}
\runningheadrule
\runningfootrule

\printanswers

\qformat{{\large\bf \thequestion. \thequestiontitle}\hfill(\thepoints)}
\boxedpoints

\title{Weekly Challenge 11: Mathematical Induction}
\author{CS/MATH 113 Discrete Mathematics}
\date{Spring 2024}

\begin{document}
\maketitle

\begin{questions}
  \titledquestion{Pyramid Scheme}[10]
    Suppose we have circular coins, a lot of them and all of the same dimension, and we were to make a hexagonal pyramid out of them as follows. The top layer has $1$ coin. The second layer has $7$ coins arranged in a hexagon with side length of two coins (see picture below). The third layer has $19$ coins in the same hexagonal pattern but with side length of three coins, and so on.

    Use mathematical induction to prove that a pyramid created in this manner and constituting $n$ layers requires in total $n^3$ coins.
  \begin{figure}[h!]
    \centerline{\includegraphics{layer1}}
    \caption{Top layer as seen from above.}
    \label{layer1}
  \end{figure}
  \begin{figure}[h!]
    \centerline{\includegraphics{layer2.png}}
    \caption{The second layer as seen from above. Note that this layer has $7$ coins and forms a hexagon with side length of two coins.}
    \label{layer2}
  \end{figure}
  \newpage
  \begin{figure}[h!]
    \centerline{\includegraphics{layer3.png}}
    \caption{The third layer with $19$ coins forming a hexagon with side length of three coins.}
    \label{layer3}
  \end{figure}

  Note: All figures are taken from \href{https://www.geogebra.org/m/cnqdjcph}{beckykwarren's geogebra page}.

  \begin{solution}
    % Enter your solution here.
    \begin{proof}
      let P(n) be the statement: $\sum^{n}_{i=1}3i^2-3i+1 = n^3$ where $n \geq 1$\\
      Base case: for n = 1\\
      $\sum^{1}_{i=1}3i^2-3i+1 = 1^3 = 1$\\
      hence P(1) is true\\\\
      Inductive step:\\
      consider that P(k) is true\\
      that is the inductive hypothesis: $\sum^{k}_{i=1}3i^2-3i+1 = k^3$\\
      lets explore P(k+1):\\
      we need to prove that $\sum^{k+1}_{i=1}3i^2-3i+1 = (k+1)^3$ \\
      LHS:\\
      $\sum^{k+1}_{i=1}3i^2-3i+1 = \sum^{k}_{i=1}3i^2-3i+1 + (3(k+1)^2-3(k+1)+1)$\\
      $=k^3 + (3(k+1)^2-3(k+1)+1)$ using the Inductive hypothesis\\
      $= k^3+3k^2+6k+3-3k-3+1$\\
      $= k^3+3k^2+3k+1$\\
      $= (k+1)^3$ 


    \end{proof}
  \end{solution}


\end{questions}

\end{document}
%%% Local Variables:
%%% mode: latex
%%% TeX-master: t
%%% End:
