\documentclass[a4paper]{exam}

\usepackage{geometry}
\usepackage{amsmath}
\usepackage{amsfonts}

\usepackage{amsmath}
\usepackage{amsthm}

\title{Problem Set 11: Mathematical Induction}
\author{CS/MATH 113 Discrete Mathematics}
\date{Spring 2024}

\boxedpoints

\printanswers

\begin{document}
\maketitle

\begin{questions}
\question Prove the following using mathematical induction on $n\in\mathbb{Z}$ where $n>0$.
  \begin{parts}
  \part  $\sum_{r=1}^{n} r(r + 1) = \frac{1}{3} \cdot n(n + 1)(n + 2)$
    \begin{solution}
      \begin{proof}
      Let P(n) be the statement: $\sum_{r=1}^{n} r(r + 1) = \frac{1}{3} \cdot n(n + 1)(n + 2)$\\
      Base step: For n = 1,\\
      $\sum_{r=1}^{1} r(r+1) = 1 \cdot (1+1) = 2$,\\
      $\frac{1}{3} \cdot 1 \cdot (1+1) \cdot (1+2) = \frac{1}{3} \cdot 1 \cdot 2 \cdot 3 = 2$,\\
      which shows that P(1) is true.\\
      
      Inductive step: Assume that P(k) is true for some positive integer k,\\
      that is, $\sum_{r=1}^{k} r(r + 1) = \frac{1}{3} \cdot k(k + 1)(k + 2).\\\\$
      Let us explore P(k+1):\\
      $\sum_{r=1}^{k+1} r(r+1) = \frac{1}{3} \cdot (k+1)(k + 2)(k + 3)$\\
      LHS:\\
      $\sum_{r=1}^{k+1} r(r+1) = \sum_{r=1}^{k} r(r + 1) + (k+1)(k+2)$

      by the inductive hypothesis, this is\\
      $=\frac{1}{3} \cdot k(k+1)(k+2) + (k+1)(k+2)$,\\
      $=(k+1)(k+2) \left( \frac{1}{3}k + 1 \right)$,\\
      $=\frac{1}{3}(k+1)(k+2)(k+3)$
      \end{proof}

      





    \end{solution}
    
  \part $1^5 + 2^5 + 3^5 + \ldots + n^5 = \frac{1}{12}  n^2 (n + 1)^2 (2n^2 + 2n - 1)$
    \begin{solution}
      % Enter your solution here.
      \begin{proof}
      Let P(n) be the statement: $ \sum_{i=1}^{n} i^5 = \frac{1}{12}  n^2 (n + 1)^2 (2n^2 + 2n - 1)$\\
      Base step: For n = 1,\\
      $ \sum_{i=1}^{1} 1^5 = \frac{1}{12}  1^2 (1 + 1)^2 (2^2 + 2 - 1) = 1$\\
      which shows that P(1) is true.\\
      
      Inductive step: Assume that P(k) is true for some positive integer k,\\
      $ \sum_{i=1}^{k} i^5 = \frac{1}{12}  k^2 (k + 1)^2 (2k^2 + 2k - 1)$\\
      Let us explore P(k+1):\\
      $\sum_{i=1}^{k+1} i^5 = \sum_{i=1}^{k} i^5 + (k+1)^5$\\
      by the inductive hypothesis, this is\\
      $\sum_{i=1}^{k+1} i^5 = \frac{1}{12} k^2 (k + 1)^2 (2k^2 + 2k - 1) + (k+1)^5$\\
      $= \cdots$\\
      $= \sum_{i=1}^{k+1} i^5 = \frac{1}{12} (k+1)^2 (k + 2)^2 (2(k+1)^2 + 2(k+1) - 1)$
      \end{proof}
    
      

    \end{solution}

  \part $5^{2n+1} + 2^{2n+1}$ is divisible by 7.
    \begin{solution}
      % Enter your solution here.
      \begin{proof}
      Let P(n) be the statement: $5^{2n+1} + 2^{2n+1}$ is divisible by 7.\\
      Base step: For n = 0,\\
      $5^{2 \cdot 0+1} + 2^{2 \cdot 0+1} = 5^1 + 2^1 = 5 + 2 = 7$,\\
      which is clearly divisible by 7, so $P(0)$ is true.\\
      
      Inductive step: Assume that $P(k)$ is true for some non-negative integer k,\\
      that is, $5^{2k+1} + 2^{2k+1}$ is divisible by 7.\\
      Now consider $P(k+1)$:\\
      $5^{2(k+1)+1} + 2^{2(k+1)+1} = 5^{2k+3} + 2^{2k+3}$,\\
      $25 \cdot 5^{2k+1} + 4 \cdot 2^{2k+2}$,\\
      $21 \cdot 5^{2k+1} + 4(5^{2k+1} + 2^{k+1})$,\\
      since we know that $5^{2k+1} + 2^{k+1} $is the inductive hypothesis and 21 is divisible by 7, P(k+1) is also divisible by 7
      
      \end{proof}
      
      
      
    \end{solution}
    
  \part $a^{2^n} - 1$ is divisible by $8 \times 2^{n-1}$ for all odd integers $a$.
    \begin{solution}
      % Enter your solution here.
      \begin{proof}
        let P(n) be the statement:  $a^{2^n} - 1$ is divisible by $8 \times 2^{n-1}$\\
        Base step: For n = 0\\
        $a^{2^1} - 1 = a^{2} - 1$ is divisible by $8 \times 2^{0} = 8$\\
        since a is odd it can be written as $2m + 1$\\
        $(2m+1)^{2} - 1 = 4m^{2} + 4m = 4m(m+1)$\\
        since 4m(m+1) is divisible by 8, P(1) is true\\
        \\
        Inductive step: Assume that P(k) is true for positive integer k,\\
        that is, $a^{2^k} - 1$ is divisible by $8 \times 2^{k-1}$.\\
        Lets explore P(k+1):\\
        we have to prove that $a^{2^{k+1}} - 1$ is divisible by $8 \times 2^{k}$\\
        $a^{2^k\cdot2^1} - 1 = a^{(2^k)^2} - (1)^2 = (a^{2^k} - 1)(a^{2^k} + 1)$\\
        using the inductive hypothesis $(a^{2^k} - 1)$ is divisible by $8 \times 2^{k-1}$,
        and $a^{2^k} + 1$ is even so it is divisible by 2\\
        Therefore the whole term $(a^{2^k} - 1)(a^{2^k} + 1)$ is divisible by $8 \times 2^{k-1} \times 2$\\
        since $8 \times 2^{k-1} \times 2 = 8 \times 2^{k}$, P(k+1) is proved. 


      \end{proof}



    \end{solution}
    
  \part if $A_1, A_2,\ldots,A_n$ and $B$ are sets, then
    \[
      (A_1-B)\cup(A_2-B)\cup\ldots\cup(A_n-B)=(A_1\cup A_2\cup\ldots \cup A_n)-B
    \]
    \begin{solution}
      % Enter your solution here.
      \begin{proof}
        let P(n): $(A_1-B)\cup(A_2-B)\cup\ldots\cup(A_n-B)=(A_1\cup A_2\cup\ldots \cup A_n)-B$
        Base step: for n = 1\\
        $(A_1-B)=(A_1-B)$\\
        hence, P(1) is true\\\\
        Inductive step: Assume that P(k) is true for some sets $A_1,A_2,\cdots,A_k $ and $B$\\
        that is, $(A_1-B)\cup(A_2-B)\cup\ldots\cup(A_k-B)=(A_1\cup A_2\cup\ldots \cup A_k)-B$\\
        lets explore P(k+1):\\
        $(A_1\cup A_2\cup\ldots \cup A_k)-B \cup (A_{k+1} - B) = (A_1-B)\cup(A_2-B)\cup\ldots\cup(A_k-B) \cup (A_{k+1} - B)$\\


      \end{proof}
    \end{solution}
  \end{parts}

\question For which non-negative integers $n$ does the property $n^2 \le n!$ hold? Prove your answer using mathematical induction.
  \begin{solution}
    % Enter your solution here
    Base Case: For \( n = 0 \), the inequality becomes \( 0^2 \leq 0! \), which simplifies to \( 0 \leq 1 \). This is true.

Inductive Step: Assume the inequality holds for some non-negative integer \( k \), i.e., \( k^2 \leq k! \). We need to show that \( (k+1)^2 \leq (k+1)! \).

Starting with the left side of the inequality:
\begin{align*}
(k+1)^2 &= k^2 + 2k + 1 \\
&\leq k! + 2k + 1 \quad \text{(by the inductive hypothesis)} \\
&\leq k! + 2k! \quad \text{(since \( k! \geq 2k \) for \( k \geq 2 \))} \\
&= k!(1 + 2) \\
&= 3k! \\
&\leq k!(k + 1) \quad \text{(since \( 3 \leq k + 1 \) for \( k \geq 2 \))} \\
&= (k+1)!
\end{align*}

Thus, the inequality \( (k+1)^2 \leq (k+1)! \) holds for \( k \geq 2 \).
  \end{solution}

\question We are familiar with the Fibonacci numbers, defined as $f_{n+1} = f_n + f_{n-1}, f_0=1, f_1=1$ for all $n \geq 1$. The Fibonacci numbers turn out to have several interesting properties. Prove each of the following by mathematical induction.
  \begin{parts}
  \part $f_1 + f_2 + f_3 + \ldots + f_n = f_{n+2} - 1$.
    \begin{solution}
      % Enter your solution here.
      Base Case: For $n=1$, we have $f_1 = 1$ and $f_{1+2} - 1 = f_3 - 1 = 2 - 1 = 1$. Hence, the property holds.

Inductive Step: Assume the property holds for some $n=k$, i.e., $f_1 + f_2 + \ldots + f_k = f_{k+2} - 1$. We need to show it holds for $n=k+1$.

\begin{align*}
f_1 + f_2 + \ldots + f_k + f_{k+1} &= (f_{k+2} - 1) + f_{k+1} \\
&= f_{k+2} + f_{k+1} - 1 \\
&= f_{k+3} - 1
\end{align*}

Thus, the property holds for $n=k+1$.

    \end{solution}

  \part $f_2 + f_4 + f_6 + \ldots + f_{2n} = f_{2n+1} - 1$.
    \begin{solution}
      % Enter your solution here.
      Base Case: For $n=1$, we have $f_2 = 1$ and $f_{2\cdot1+1} - 1 = f_3 - 1 = 2 - 1 = 1$. Hence, the property holds.

Inductive Step: Assume the property holds for some $n=k$, i.e., $f_2 + f_4 + \ldots + f_{2k} = f_{2k+1} - 1$. We need to show it holds for $n=k+1$.

\begin{align*}
f_2 + f_4 + \ldots + f_{2k} + f_{2(k+1)} &= (f_{2k+1} - 1) + f_{2k+2} \\
&= f_{2k+1} + f_{2k+2} - 1 \\
&= f_{2k+3} - 1
\end{align*}

Thus, the property holds for $n=k+1$
    \end{solution}
  \end{parts}

\question A \textit{convex polygon} is a polygon in which all the interior angles are smaller than $\pi$ radians. An $n$-gon is a polygon with $n$ sides. Give a proof by mathematical induction that the sum of the interior angles of a convex $n$-gon is $(n-2)\pi$ where $n\ge 3$. 

  \textit{Hint}: A convex $n$-gon can be divided into a convex $(n-1)$-gon and a triangle.
  \begin{solution}
    % Enter your solution here.
    \begin{proof}
      P(n): An n-gon is polygon with n sides interior angles of a convex n-gon is $(n - 2) \pi$ where $n \ge 3$\\
      Base case: for n = 3\\
      A 3 sided polygon is a triangle and we know that the sum of it's interior anges is $\pi$\\
      $(3-2)\pi = \pi$, hence P(3) is true\\\\
      Inductive step: Assume that P(k) is true,\\
      that is, The sum of the interior angles of a convex k-gon is $(k - 2)\pi$\\
      lets explore P(k+1): we have to prove that sum of interior angles of a convex (k+1)-gon is $(k - 1) \pi$\\
      A convex (k + 1)-gon can be divided into a convex k-gon and a triangle. The sum of
      the interior angles of a convex (k + 1)-gon is the sum of the interior angles of the convex
      k-gon plus the sum of the interior angles of the triangle,\\
      that is, $(k-2)\pi + \pi$ which is equals to $(k-1)\pi$

    \end{proof}
  \end{solution}
  

\end{questions}
\end{document}
%%% Local Variables:
%%% mode: latex
%%% TeX-master: t
%%% End: