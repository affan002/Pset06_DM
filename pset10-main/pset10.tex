\documentclass[a4paper]{exam}

\usepackage{geometry}
\usepackage{amsmath}
\usepackage{amsfonts}

\newcommand\Z{\ensuremath{\mathbb{Z}}}
\newcommand\R{\ensuremath{\mathbb{R}}}
\newcommand\Q{\ensuremath{\mathbb{Q}}}

\title{Problem Set 10: Cardinality Of Sets}
\author{CS/MATH 113 Discrete Mathematics}
\date{Spring 2024}

\boxedpoints

\printanswers

\begin{document}
\maketitle

\begin{questions}
\question 
  Show that the two given sets have equal cardinality by describing a bijection from one to the other. Describe your bijection with a formula (not as a table). You need not include a proof of bijection.
  \begin{parts}
  \part The set of even integers and the set of odd integers
    \begin{solution}
      % Enter your solution here.
    \end{solution}
    
  \part $\Z$ and $S = \{..., \frac{1}{8}, \frac{1}{4}, \frac{1}{2}, 1, 2, 4, 8, 16, ...\}$
    \begin{solution}
      % Enter your solution here.
    \end{solution}

  \part $\Z^+$ and $S = \{2^n : n \in \Z^+\}$
    \begin{solution}
      % Enter your solution here.
    \end{solution}
    
  \part $A = \{3k : k \in \Z\}$ and $B = \{7k : k \in \Z\}$
    \begin{solution}
      % Enter your solution here.
    \end{solution}
    
  \part $\R$ and $S=\{x\in\R\mid x>0 \}$
    \begin{solution}
      % Enter your solution here.
    \end{solution}
    
  \part $\R$ and $S=\{x\in\R\mid x>\sqrt{2} \}$
    \begin{solution}
      % Enter your solution here.
    \end{solution}
  \end{parts}

\question Prove or disprove each of the following statements.
  \begin{parts}
  \part If $A = \{X \subseteq \Z^+ \mid X \text{ is finite}\}$, then $|A| = \aleph_0$.
    \begin{solution}
      % Enter your solution here.
    \end{solution}
  \part The set $A = \{(m,n) \in \Z^+ \times \Z^+ \mid m \leq n\}$ is countably infinite.

    \begin{solution}
      % Enter your solution here.
    \end{solution}
  \part The set $\Z \times \Q$ is countably infinite.
    \begin{solution}
      % Enter your solution here.
    \end{solution}

  \part If $A \subseteq B$ and there is an injection $g : B \to A$, then $|A| = |B|$.
    \begin{solution}
      % Enter your solution here.
    \end{solution}
    
  \part  If $|A| = |B|$ and $|B| = |C|$, then $|A| = |C|$.
    \begin{solution}
      % Enter your solution here.
    \end{solution}
    
  \part If $A$ and $B$ are sets with $|A| = |B|$, then $|\mathcal{P}(A)| = |\mathcal{P}(B)|$.
    \begin{solution}
      % Enter your solution here.
    \end{solution}
  \end{parts}
  
\end{questions}
\end{document}
%%% Local Variables:
%%% mode: latex
%%% TeX-master: t
%%% End: