\documentclass[a4paper]{exam}


\usepackage[a4paper]{geometry}
\usepackage{amsfonts, amsmath, amsthm}
\usepackage[framemethod=TikZ]{mdframed}

\newcommand\Z{\ensuremath{\mathbb{Z}}}

\title{Problem Set 12: Strong Induction}
\author{CS/MATH 113 Discrete Mathematics}
\date{Spring 2024}

\boxedpoints

\printanswers

\begin{document}
\maketitle

In solving the problems below, note that mathematical induction is just a special case of strong induction.

\begin{questions}

\question Use strong induction to prove each of the statements given below.
  \begin{parts}
  \part Every integer \( n \geq 2 \) is either prime or a product of primes.\\
    \textit{Hint}: Consider the cases where \( k+1 \) is prime and where it is composite.
   \begin{solution}
    % Enter your solution here.
    \begin{proof}
      let P(n): n can be written as the product primes or it is a prime \\
      Basis step: for n=2\\
      P(2) is true as 2 is itself a prime number\\\\
      Incuctive step: consider that P(j) is true for $2 \leq j \leq k$\\
      exploring P(k+1): there would be two cases, either k+1 is a prime number or it is a composite number\\
      In case of it being a prime number, P(k+1) is true as it is itself a prime\\
      In the other case, we can say that k+1 can be written as the product of two integers a and b, such that $2\leq a\leq b < k$\\
      then using the inductive hypothesis we can say that both both  a and b are either prime or they can be written as the product of primes\\
      Therefore, k+1 can be written as product of primes, which are mainly the prime factors of the integer a and b.
    \end{proof}
   \end{solution}

 \part \( n! \ge n \) for all integers, \( n \geq 1 \) using strong induction.
\begin{solution}
    % Enter your solution here.
    \begin{proof}
      let P(n): \( n! \ge n \)\\
      Basis step: for n =1\\
      $1! \geq 1$\\
      Therefore, P(1) is true\\\\
      Inductive step: consider p(j) is true for  $1 \leq j \leq k$\\
      exploreing P(k+1): \\
      we have to prove that $(k+1)! \geq k+1$\\
      using the inductive hypothesis we have,  $k! \ge k$\\
      by multiplying k+1 on both sides we get $(k+1)! \geq k(k+1)$, \\
      since $k(k+1) \geq k+1$, we can say that $(k+1)! \geq k+1$
    \end{proof}
  \end{solution}

\part  \( \Z^n \) is countable for all integers \( n \geq 1 \).\\
  \textit{Hint}: You can refer to known results and proofs to simplify your proof.\\
  \textit{Note}: Recall that \( \Z^n = \underbrace{\Z\times\Z\times\ldots\times\Z}_{n \text{ times}} \)
  \begin{solution}
    % Enter your solution here.
    let P(n): $\Z^n$ is countable\\
    Basis step: for n=1, we know that $\Z$ P(1) is true as the set of integers is countable\\\\
    Inductive step: Assume that P(j) is true for $1 \leq j \leq k$\\
    lets explore P(k+1)\\
    $\Z^{k+1} = \Z^k \times \Z$\\
    using the inductive hypothesis, $\Z^k$ and $\Z$ are countable\\
    The cartesian product $\Z^k \times \Z$ is also countable\\
    Thus P(k+1) is true
  \end{solution}
\end{parts}

\question The game of Mini-Nim is defined as follows: Some positive number of sticks are placed on the ground. Two players take turns removing one, two, or three sticks. The player to remove the last one loses.

  Use strong induction to show that: The second player has a winning strategy only if the number of sticks equals \( 4k + 1 \) for some integer \( k \ge 0 \).\\
  \textit{Note}: You will have to consider all cases for the number of sticks: \( 4k+1, 4k+2, 4k+3, \) and \( 4k+4 \), and show that Player 2 can win in one case, and that Player 1 can win in all the other cases.

  \begin{solution}
    % Enter your solution here.
    \begin{proof}
      let P(n): If $n = 4k + 1$, for some $k \in N$, then the second player has a winning strategy otherwise, the first player has a winning strategy.
      Base case: n = 1. The first player has no choice but to remove 1 stick and lose, which is
what the theorem says for this case.\\\\
      Inductive step: Assume that P(j) is true for $1 \leq j \leq k$, where $k > 1$\\
      Consider the case when there are $n + 1$ sticks.
\begin{itemize}
    \item If $n + 1$ is of the form $4k + 1$, then the second player can win immediately by taking the appropriate number of sticks to reduce the remaining pile to $4k + 1$ sticks (which is the winning position).
    \item Otherwise, we have three possibilities:
    \begin{itemize}
        \item $n + 1 = 4k + 2$: The second player can take 1 stick to leave $4k + 1$ sticks for the first player. By the inductive hypothesis, the first player will lose.
        \item $n + 1 = 4k + 3$: The second player can take 2 sticks to leave $4k + 1$ sticks for the first player. Again, by the inductive hypothesis, the first player will lose.
        \item $n + 1 = 4k + 4$: The second player can take 3 sticks to leave $4k + 1$ sticks for the first player. Once more, by the inductive hypothesis, the first player will lose.
    \end{itemize}
\end{itemize}

Conclusion:
By strong induction, we have shown that the second player has a winning strategy if and only if the number of sticks equals $4k + 1$ for some integer $k \geq 0$.

    \end{proof}
  \end{solution}
  
  \question 
  Consider the following proof by strong induction that a class with \( n \geq 8 \) students can be divided into groups of 4 or 5.

  \begin{mdframed}
    \begin{proof} The proof is by strong induction.

      Let \( P(n) \) be the proposition that a class with \( n \) students can be divided into teams of 4 or 5.

      First, we prove that \( P(n) \) is true for \( n = 8, 9, \) or \( 10 \) by showing how to break classes of these sizes into groups of 4 or 5 students:
      \begin{align*}
8  &= 4 + 4 \\
9  &= 4 + 5 \\
10 &= 5 + 5 
      \end{align*}
Next, we must show that \( P(8), \ldots, P(n) \) imply \( P(n + 1) \) for all \( n \geq 10 \). Thus, we assume that \( P(8), \ldots, P(n) \) are all true and show how to divide up a class of \( n + 1 \) students into groups of 4 or 5. We first form one group of 4 students. Then we can divide the remaining \( n-3 \) students into groups of 4 or 5 by the assumption \( P(n-3) \). This proves \( P(n+1) \), and so the claim holds by induction.
\end{proof}
\end{mdframed}

a)
This proof contains a critical logical error. (In fact, the claim is false!) Identify the first sentence in the proof that does not follow and explain what went wrong.
  \begin{solution}
    The basis step doesn’t include P(11) and so the argument is not established
correctly as P(11) cannot be proven in this case.

      
  \end{solution}
b)
Provide a correct strong induction proof that a class with \( n \geq 12 \) students can be divided into groups of 4 or 5.
  \begin{solution}
    P(n): n students can be divided into groups of 4 and 5\\\\
    Basis step: we can divide 12, 13, 14, 15, 16 students into 3 - 4 people group, 2 - 4 people and 1 - 5 people groups,     
    2-5 people and 1-4 people groups, 3-5 people groups, 4-4 people groups, respectively.\\\\
    Inductive step:\\
    Assume P(j) for $12 \leq j \leq k$, where $k \geq 16$\\
    We need to show that P(k+1) is true\\
    Using the inductive hypothesis we know that $k-3$ student can be divided into groups of 4 and 5\\
    We can get $k+1$ students by adding a group of 4 people to k-3 students\\
    Hence P(k+1) is true
  \end{solution}

\end{questions}
\end{document}
%%% Local Variables:
%%% mode: latex
%%% TeX-master: t
%%% End: