\documentclass[a4paper]{exam}

\usepackage{amsmath}
\usepackage{geometry}

\header{CS/MATH 113}{PSet 03}{Spring 2024}
\footer{}{Page \thepage\ of \numpages}{}
\runningheadrule
\runningfootrule

\title{Problem Set 03: Propositional Logic and Equivalences}
\author{CS/MATH 113 Discrete Mathematics}
\date{Spring 2024}

\boxedpoints

\printanswers

\begin{document}
\maketitle

\begin{questions}

  \question Show that the following are logically equivalent without using truth tables:
  \begin{parts}
    \part $(p \Rightarrow r) \lor (q \Rightarrow r) \equiv (p \land q) \Rightarrow r$

    \begin{solution}
      % Enter your solution here.
      \\$(\neg p \lor r) \lor (\neg q \lor r)$ Conditional disjunction \\
      $(\neg p \lor \neg q) \lor (r \lor r)$ Associative law\\
      $\neg(p \land q) \lor r$ Idempotent and demorgan's law\\
      $(p \land q) \rightarrow r$ conditional disjunction
    \end{solution}
    \part 
    $\neg \left[ \neg \left[\left(p \lor q \right) \land r \right] \lor \neg q \right] \equiv q \land r $

    \begin{solution}
      % Enter your solution here.
      \\$\neg \left[\neg(p \lor q) \lor \neg r \lor \neg q \right]$ Demorgan's law\\
      $(p \lor q) \land r \land q$ Demorgan's law\\
      $\left[q \land (p \lor q) \right] \land r$ Associative law\\
      $\left[(q \land p) \lor (q \land q)\right] \land r$ Distributive law\\
      $\left[q \lor (q \land p) \right] \land r$ Idempotent law\\
      $q \land r$ Absorption law
    \end{solution}
    \part  $(p \lor q \lor r) \land (p \lor t \lor \neg q) \land (p \lor \neg t \lor r)  \equiv p \lor \left[ r \land \left(t \lor \neg q \right)\right]$

    \begin{solution}
      % Enter your solution here.


    \end{solution}
  \end{parts}
  
  \question Use Truth tables to see if the following statements are logically equivalent: 
  \begin{parts}
    \part $p \Rightarrow (q \lor r) \equiv (q \Rightarrow p) \land (p \Rightarrow r)$

    \begin{solution}\\
      % Enter your solution here.
      \begin{tabular}{|ccc|c|c|c|c|c|}
        \hline
        p & q & r & $q \lor r$ & $p \rightarrow (q \lor r)$ & $q \rightarrow p$ & $p \rightarrow r$ & $(q \rightarrow p) \land (p \rightarrow r)$ \\ \hline
        T & T & T & T & T & T & T & T\\
        T & T & F & T & T & T & F & F\\
        T & F & T & T & T & T & T & T\\
        T & F & F & F & F & T & F & F\\
        F & T & T & T & T & F & T & F\\
        F & T & F & T & T & F & T & F\\
        F & F & T & T & T & T & T & T\\
        F & F & F & F & T & T & T & T\\ \hline
      \end{tabular}\\
      since the truth value of the statements donot match, they are not equivalent.
    \end{solution}
    \part 
    $(p \lor q) \Rightarrow r \equiv (p \Rightarrow r) \land (q \Rightarrow r)$

    \begin{solution}\\
      % Enter your solution here.
      \begin{tabular}{|ccc|c|c|c|c|c|}
        \hline
        p & q & r & $p \lor q$ & $p \rightarrow r$ & $q \rightarrow r$ & $(p \lor q) \rightarrow r$ & $(p \rightarrow r) \land (q \rightarrow r)$ \\ \hline
        T & T & T & T & T & T & T & T\\
        T & T & F & T & F & F & F & F\\
        T & F & T & T & T & T & T & T\\
        T & F & F & T & F & T & F & F\\
        F & T & T & T & T & T & T & T\\
        F & T & F & T & T & F & F & F\\
        F & F & T & F & T & T & T & T\\
        F & F & F & F & T & T & T & T\\ \hline
      \end{tabular}\\
      Since the truth values for both Propositional statements are same, they are logically equivalent.
    \end{solution}
    \part  $p \Rightarrow (q \lor r) \equiv \neg r \Rightarrow (p \Rightarrow q)$

    \begin{solution}\\
      % Enter your solution here.
      \begin{tabular}{|ccc|c|c|c|c|c|}
        \hline
        p & q & r & $q \lor r$ & $\neg r$ & $p \rightarrow q$ & $p \rightarrow (q \lor r)$ & $\neg r \rightarrow (p \rightarrow q) $ \\ \hline
        T & T & T & T & F & T & T & T\\
        T & T & F & T & T & T & T & T\\
        T & F & T & T & F & F & T & T\\
        T & F & F & F & T & F & F & F\\
        F & T & T & T & F & T & T & T\\
        F & T & F & T & T & T & T & T\\
        F & F & T & T & F & T & T & T\\
        F & F & F & F & T & T & T & T\\ \hline
      \end{tabular}\\
      Since the truth values of the two Propositional are both the same, they are logically Equivalent.
    \end{solution}
  \end{parts}

  \question Express the negation of each of the following statements in natural language using De Morgan’s laws.
  \begin{parts}
  \part Graduates take a job in industry or go to graduate school or start their own ventures.
  \begin{solution}
    % Enter your solution here.
    Graduates will not take a job in industry and won't go to graduate school and won't start their own ventures.
  \end{solution}
  \part First year students know python and calculus.
  \begin{solution}
    % Enter your solution here.
    First year students don't know python or they don't know calculus
  \end{solution}
  \part Horizon is new and bright.
  \begin{solution}
    % Enter your solution here.
    Horizon is not new or not bright.
  \end{solution}
\end{parts}

\question Determine whether each of the following compound propositions is satisfiable.
\begin{parts}
  \part $(p\lor \lnot q)\land(\lnot p\lor q)\land(\lnot p\lor\lnot q)$
  \begin{solution}
    % Enter your solution here.
    When both the truth value of both p and q are False, the compound proposition is True. Hence it is satisfiable because of the statement being a contingency.
  \end{solution}
  \part $(p\implies q)\land(p\implies\lnot q)\land (\lnot p\implies q)\land(\lnot p\implies\lnot q)$
  \begin{solution}
    % Enter your solution here.
    This compound proposition can be also be written as\\ $ \left[(\neg p \lor q)\land(\neg p \lor \lnot q) \land(\ p\lor\lnot q)\right] \land (p \lor q)$ \\ The statement to left is the same as the one in last part for which we reasoned that it is only True when both p and q are False, whereas $p \lor q$ is True when atleast of one of them is True. This causes the conditions to be contradictory thus the whole compound proposition is a contradiction, hence unsatisfiable.
  \end{solution}
  \part $(p \iff q)\land(\lnot p \iff q)$
  \begin{solution}
    % Enter your solution here.
    There is no assignment of truth values to \(p\) and \(q\) that makes the whole propositional statement True, making this statement a contradiction. Thus it is not satisfiable.
  \end{solution}
\end{parts}
\end{questions}
\end{document}
%%% Local Variables:
%%% mode: latex
%%% TeX-master: t
%%% End: