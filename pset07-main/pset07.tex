\documentclass[a4paper]{exam}


\usepackage{amsfonts,amsmath,amsthm}
\usepackage[a4paper]{geometry}
\usepackage{hyperref}
\usepackage{xcolor}

\theoremstyle{definition}
\newtheorem{definition}{Definition}

\newcommand\Z{\ensuremath{\mathbb{Z}}}

\title{Problem Set 07: Proof Methods}
\author{CS/MATH 113 Discrete Mathematics}
\date{Spring 2024}

\boxedpoints

\printanswers

\begin{document}
\maketitle

For each question below, clearly write the statement to prove or disprove so that its logical structure is evident. Provide formal proofs. See \href{https://www.overleaf.com/learn/latex/Theorems_and_proofs#Proofs}{this guide} for typesetting proofs in \LaTeX.

If your proof is exceeding 10 lines, you are probably on the wrong track.

The following definitions may prove helpful when attempting the problems.

\begin{definition}[Prime and composite numbers]
A natural number (0, 1, 2, 3, 4, 5, 6, etc.) is called a \textit{prime number} (or a \textit{prime}) if it is greater than 1 and cannot be written as the product of two smaller natural numbers. The numbers greater than 1 that are not prime are called \textit{composite} numbers.  
\end{definition}

\begin{definition}[Even and odd numbers]
An \textit{even number} is an integer of the form $x=2k$ where $k$ is an integer; an \textit{odd number} is an integer of the form $x=2k+1$.  
\end{definition}

\begin{definition}[Parity]
The \textit{parity of a number} is its property of being even or odd.
\end{definition}

\begin{definition}[Rational number]
A \textit{rational number} can be written as $\frac{p}{q}$ where $p$ and $q$ are integers and $q\neq 0$. A number that is not rational is \textit{irrational}.
\end{definition}

\begin{questions}
  
\question
  Show through contraposition: If $x^2 - 6x + 5$ is even, then $x$ is odd.
  \begin{solution}
    % Enter your solution here.
  \end{solution}

\question Provide a counterexample to disprove: If $n$ is an integer and $n^2$ is divisible by 4, then $n$ is divisible by 4. Explain why it is a counterexample.

  \begin{solution}
    % Enter your solution here.
  \end{solution}
  
\question Prove using contradiction that $\sqrt{2}$ is irrational.

  \begin{solution}
    % Enter your solution here.
  \end{solution}
  
\fullwidth{For each of the following problems, clearly mention the proof method that you employ.}

\question Prove that for $n\in\Z$, $n$ is odd if and only if $5n + 6$ is odd.

  \begin{solution}
    % Enter your solution here.
  \end{solution}

\question Prove or disprove: The sum of a rational and an irrational number is a rational number.

  \begin{solution}
    % Enter your solution here.
    \end{solution}
  
\question Prove or disprove that for $(x^2 - y^2) \mod 4 \neq 2$ where $x$ and $y$ are integers.\\
  \textit{Hint}: a) Consider the different cases of parities of $x$ nd $y$. b) Use the method of \textit{proof by cases} and apply a proof \textit{without loss of generality} described in Section 1.8.2 in the book.

  \begin{solution}
    % Enter your solution here.
  \end{solution}

\question 
  Prove or disprove that $2^n + 1$ is prime for every $n\in\Z^+$.

  \begin{solution}
    % Enter your solution here.
  \end{solution}
\end{questions}
\end{document}
%%% Local Variables:
%%% mode: latex
%%% TeX-master: t
%%% End: