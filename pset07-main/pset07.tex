\documentclass[a4paper]{exam}


\usepackage{amsfonts,amsmath,amsthm}
\usepackage[a4paper]{geometry}
\usepackage{hyperref}
\usepackage{xcolor}

\theoremstyle{definition}
\newtheorem{definition}{Definition}

\newcommand\Z{\ensuremath{\mathbb{Z}}}

\title{Problem Set 07: Proof Methods}
\author{CS/MATH 113 Discrete Mathematics}
\date{Spring 2024}

\boxedpoints

\printanswers

\begin{document}
\maketitle

For each question below, clearly write the statement to prove or disprove so that its logical structure is evident. Provide formal proofs. See \href{https://www.overleaf.com/learn/latex/Theorems_and_proofs#Proofs}{this guide} for typesetting proofs in \LaTeX.

If your proof is exceeding 10 lines, you are probably on the wrong track.

The following definitions may prove helpful when attempting the problems.

\begin{definition}[Prime and composite numbers]
A natural number (0, 1, 2, 3, 4, 5, 6, etc.) is called a \textit{prime number} (or a \textit{prime}) if it is greater than 1 and cannot be written as the product of two smaller natural numbers. The numbers greater than 1 that are not prime are called \textit{composite} numbers.  
\end{definition}

\begin{definition}[Even and odd numbers]
An \textit{even number} is an integer of the form $x=2k$ where $k$ is an integer; an \textit{odd number} is an integer of the form $x=2k+1$.  
\end{definition}

\begin{definition}[Parity]
The \textit{parity of a number} is its property of being even or odd.
\end{definition}

\begin{definition}[Rational number]
A \textit{rational number} can be written as $\frac{p}{q}$ where $p$ and $q$ are integers and $q\neq 0$. A number that is not rational is \textit{irrational}.
\end{definition}

\begin{questions}
  
\question
  Show through contraposition: If $x^2 - 6x + 5$ is even, then $x$ is odd.
  \begin{solution}
    % Enter your solution here.
    \begin{proof}
      We will prove the contrapositive: If $x$ is even, then $x^2 - 6x + 5$ is odd.
      
      suppose $x$ is even, which means $x = 2k$ for some integer $k$.
      
      Then $x^2 - 6x + 5 = (2k)^2 - 6(2k) + 5 = 4k^2 - 12k + 4 + 1$.
      
      We can factor out $2$ from the first three terms and get $2(2k^2 - 6k + 2) + 1$.
      
      Since $2k^2 - 6k + 2$ is an integer, we can write $2(2k^2 - 6k + 2) + 1 = 2m + 1$ for some integer $m$.
      
      Now we can see that $2m + 1$ is odd
      
      Therefore, $x^2 - 6x + 5$ is odd.
      
      We have proved the contrapositive, so we can conclude that the original statement is true.
      \end{proof}
      
  \end{solution}

\question Provide a counterexample to disprove: If $n$ is an integer and $n^2$ is divisible by 4, then $n$ is divisible by 4. Explain why it is a counterexample.

  \begin{solution}
    % Enter your solution here.
    \begin{proof}
    Let $n = 2$. Then $n^2 = 4$, which is divisible by $4$. However, $n = 2$ is not divisible by $4$, since $4$ does not divide $2$ evenly. Therefore, $n = 2$ is a counterexample to the statement.
    \end{proof}
  \end{solution}
  
\question Prove using contradiction that $\sqrt{2}$ is irrational.

  \begin{solution}
    % Enter your solution here.
    \begin{proof}
      Suppose $\sqrt{2}$ is rational, and let $\sqrt{2} = \frac{a}{b}$, where $a$ and $b$ are positive integers with no common factor. Then we have
      \[
      2b^2 = a^2.
      \]
      This implies that $a^2$ is even, and hence $a$ is even. Therefore, we can write $a = 2k$ for some positive integer $k$. Substituting this into the equation, we get
      \[
      2b^2 = (2k)^2,
      \]
      which simplifies to
      \[
      b^2 = 2k^2.
      \]
      This implies that $b^2$ is even, and hence $b$ is even. But this contradicts the assumption that $a$ and $b$ have no common factor, since they both have 2 as a factor. Therefore, our initial assumption that $\sqrt{2}$ is rational must be false, and hence $\sqrt{2}$ is irrational.
      \end{proof}
  \end{solution}
  
\fullwidth{For each of the following problems, clearly mention the proof method that you employ.}

\question Prove that for $n\in\Z$, $n$ is odd if and only if $5n + 6$ is odd.

  \begin{solution}
    % Enter your solution here.
    \begin{proof}
We will prove this using proof by equivalency.\\
odd(x): x is odd\\
even(x): x is even

First we prove that $odd(n) \implies odd(5n+6)$ using direct proof\\
Suppose $n$ is odd. Then $n = 2k + 1$ for some integer $k$. Then
\[
5n + 6 = 5(2k + 1) + 6 = 10k + 11 = 2(5k + 5) + 1 = 2m + 1
\]
We can conclude that $5n + 6$ is odd.

We now prove the contraposition of $odd(5n+6) \implies odd(n)$ which is\\ $even(n) \implies even(5n+6)$\\
Suppose $even(n)$ is True. Then $n = 2t$ for some integer $t$. Then
\[
5n + 6 = 5(2t) + 6 = 10k + 6 = 2(5t + 3) = 2s
\]
We conclude that $5n+6$ is even.\\
Since, both of the implications are proved, the biconditional is also proved.
\end{proof}

  \end{solution}

\question Prove or disprove: The sum of a rational and an irrational number is a rational number.

  \begin{solution}
    % Enter your solution here.
    \begin{proof}
    Let $r$ be a rational number and $i$ be an irrational number. Then suppose their sum $r + i$ is rational. Then we can write $\frac{p}{q} + i = \frac{a}{b}$, where $a$, $b$, $p$ and $q$ are integers and $b \neq 0$, $q\neq0$. Then we have
\[
i = \frac{aq - pb}{bq}.
\]

Since $bq \neq 0$, i is a rational number thus contradicting with our assumption that $i$ is irrational.

Therefore, we have disproved the statement by contradiction.
\end{proof}
    \end{solution}
  
\question Prove or disprove that for $(x^2 - y^2) \mod 4 \neq 2$ where $x$ and $y$ are integers.\\
  \textit{Hint}: a) Consider the different cases of parities of $x$ nd $y$. b) Use the method of \textit{proof by cases} and apply a proof \textit{without loss of generality} described in Section 1.8.2 in the book.

  \begin{solution}
    % Enter your solution here.
    \begin{proof}

There are four possible cases for the parities of $x$ and $y$:

\begin{itemize}
    \item Case 1: $x = 2k$ and $y = 2l$ for some integers $k$ and $l$.
    \item Case 2: $x = 2k+1$ and $y = 2l$ for some integers $k$ and $l$.
    \item Case 3: $x = 2k$ and $y = 2l+1$ for some integers $k$ and $l$.
    \item Case 4: $x = 2k+1$ and $y = 2l+1$ for some integers $k$ and $l$.
\end{itemize}

We will now prove each case one by one
\begin{itemize}
    \item Case 1: $x^2 - y ^2 = (2k)^2 - (2l)^2 = 4(k^2 - l^2)$. This means that $(x^2 - y^2) \mod 4 = 0$.
    \item Case 2: $x^2 - y^2 = (2k+1)^2 - (2l)^2 = 4(k^2 + k - l^2) + 1$. This means that $(x^2 - y^2) \mod 4 = 1$.
    \item Case 3: $x^2 - y^2 = (2k)^2 - (2l+1)^2 = 4(k^2 - l^2 - l) - 1$. This means that $(x^2 - y^2) \mod 4 = 3$.
    \item Case 4: $x^2 - y^2 = (2k+1)^2 - (2l+1)^2 = 4(k^2 + k - l^2 - l)$. This means that $(x^2 - y^2) \mod 4 = 0$.
\end{itemize}

Since we have covered all possible cases, we have proved that $(x^2 - y^2) \mod 4 \neq 2$ for all integers $x$ and $y$
    \end{proof}
  \end{solution}

\question 
  Prove or disprove that $2^n + 1$ is prime for every $n\in\Z^+$.

  \begin{solution}
    % Enter your solution here.
    \begin{proof}
      This can be disproved using a counterexample.\\
      Consider n = 3, for which $2^n + 1 = 9$\\
      Since 9 can be written as a product of two $3$, it is not a prime number.\\
      Therefore we have disproved the statement. 
    \end{proof}
  \end{solution}
\end{questions}
\end{document}
%%% Local Variables:
%%% mode: latex
%%% TeX-master: t
%%% End: