\documentclass[a4paper]{exam}

\usepackage{amssymb}
\newcommand{\qed}{\hfill $\square$}
\usepackage{amsmath}
\usepackage[a4paper]{geometry}

\title{Problem Set 06: Laws of Inference}
\author{CS/MATH 113 Discrete Mathematics}
\date{Spring 2024}

\boxedpoints

\printanswers

\begin{document}
\maketitle

\begin{questions}
 \question 
  Determine whether each of these arguments is valid. If an argument is correct, what rule of inference is being used? If it is not, what logical error occurs?

\begin{parts}
    \part If $n$ is a real number such that $n > 1$, then $n^2 > 1$. \\
    Suppose that $n^2 > 1$. Then $n > 1$.
     \begin{solution}
    % Enter your solution here%
    Thie argument is invalid due to the fallacy of affirming the solution. 
    consider, n = -2\\
    $n^2 > 1 \equiv T$\\
    $n>1 \equiv F$\\
    \qed 
    
    \end{solution}
    
    \part If $n$ is a real number with $n > 3$, then $n^2 > 9$. \\
    Suppose that $n^2 \leq 9$. Then $n \leq 3$.
     \begin{solution}
    % Enter your solution here.
    This argument is true by the law of contraposition\\
    p: $n > 3$\\
    q: $n^2 > 9$\\
    $p \implies q \equiv \lnot q \implies \lnot p$\\
    \qed
    \end{solution}
    
    \part If $n$ is a real number with $n > 2$, then $n^2 > 4$. \\
    Suppose that $n \leq 2$. Then $n^2 \leq 4$.
     \begin{solution}
    % Enter your solution here.
    The argument is invalid\\
    consider, n = -5\\
    $n \leq 2 \equiv T$\\
    $n^2 \leq 4 \equiv F$\\
    \qed 
     \end{solution}
   \end{parts}
   
  \question  Identify the error or errors in this argument that supposedly shows that if $\forall x(P(x) \lor Q(x))$ is true then $\forall x P(x) \lor \forall x Q(x)$ is true.
\begin{enumerate}
    \item $\forall x(P(x) \lor Q(x))$ \hfill Premise
    \item $P(c) \lor Q(c)$ \hfill Universal instantiation from (1)
    \item $P(c)$ \hfill Simplification from (2)
    \item $\forall x P(x)$ \hfill Universal generalization from (3)
    \item $Q(c)$ \hfill Simplification from (2)
    \item $\forall x Q(x)$ \hfill Universal generalization from (5)
    \item $\forall x(P(x) \lor \forall x Q(x))$ \hfill Conjunction from (4) and (6)
\end{enumerate}
  \begin{solution}\\
    % Enter your solution here.
    Step 4: since we don't know that P(x) is true for all x or not, Universal generalization cannot be applied.\\
    Step 6: since we don't know that Q(x) is true for all x or not, Universal generalization cannot be applied.\\
    Step 7: since statement 4 and 6 are incorrect, this statement cannot be made.
  \end{solution}
  
  \question Sheikh Chilly, famous for his bizarre sense of humor and love of logic puzzles, left the following clues regarding the location of the hidden treasure. The treasure can only be in one place. If the house is next to a lake, then the treasure is in the kitchen. If the house is not next to a lake or the treasure is buried under the flagpole, then the tree in the front yard is an elm and the tree in the back yard is not an oak. If the treasure is in the garage, then the tree in the back yard is not an oak.  If the treasure is not buried under the flagpole, then the tree in the front yard is not an elm. 2 The treasure is not in the kitchen. Using rules of inference, determine where the treasure is hidden. Clearly state what your propositions represent.
  \begin{solution}\\
    % Enter your solution here.
    L: The house is next to a lake. \\
    K: The treasure is in the kitchen. \\
    F: The treasure is buried under the flagpole. \\
    E: The tree in the front yard is an elm. \\
    O: The tree in the back yard is an oak. \\
    G: The treasure is in the garage. \\
    \\
    The clues can be translated into the following logical statements: \\
    1. $L \rightarrow K$ \\
    2. $(\neg L \lor F) \rightarrow (E \land \neg O)$ \\
    3. $G \rightarrow \neg O$ \\
    4. $\neg F \rightarrow \neg E$ \\
    5. $\neg K$ \\

    \begin{tabular}{|cc|c|}
      \hline
      & step & reason \\ 
      \hline
      6) & $\neg L$ & Modus tollens on 1 and 5\\
      7) & $E \land \neg O$ & Modus ponens on 2 and 6 \\
      8) & $E$ & Simplification on 7\\
      9) & $E \implies F$ & contraposition on 4\\
      10) & $F$ & Modus ponens of 8 and 9\\
      \hline
    \end{tabular}\\
    \\\\Therefore, the treasure is buried under the flagpole
  \end{solution}
  
  \question Tommy Flanagan was telling you what he ate yesterday afternoon. He tells you, “I had either popcorn or raisins. Also, if I had cucumber sandwiches, then I had soda. But I didn't drink soda or tea.” You know that Tommy is the world’s worst liar, and everything he says is false. What did Tommy eat?
Justify your answer by writing all of Tommy's statements using sentence variables (P, Q, R, S, T), taking their negations, and using these to deduce what Tommy actually ate.

  \begin{solution}\\
    % Enter your solution here.
    P: Tommy had popcorn. \\
    Q: Tommy had raisins. \\
    R: Tommy had cucumber sandwiches. \\
    S: Tommy had soda. \\
    T: Tommy had tea. \\

    Tommy's statements can be translated into the following logical statements:

    1. $P \lor Q$ \\
    2. $R \rightarrow S$ \\
    3. $\neg S \land \neg T$ \\

    Since everything Tommy says is false, we take the negation of each statement:

    1. $\neg (P \lor Q) \equiv \neg P \land \neg Q$ \\
    2. $\neg (R \rightarrow S) \equiv R \land \neg S$ \\
    3. $\neg (\neg S \land \neg T) \equiv S \lor T$ \\

    \begin{tabular}{|l|l|}
      \hline
      step & reason\\
      \hline
      4) $\neg S$ & Simplification on 2\\ 
      5) $R$ & Simplification on 2\\
      6) $T$ & Disjuntive syllogism on 3 and 4\\
      7) $R \land T$ & Conjunction of 5 and 6\\
      \hline
    
    \end{tabular}\\\\
    Therefore, Tommy ate cucumber sandwiches and drank tea.

  \end{solution}

  \question Following is a quote by Sherlock Holmes from \textit{“A Study in Scarlet”} in which he solves a murder case.
\begin{quote}
``And now we come to the great question as to the reason why. Robbery has not been the object of the murder, for nothing was taken. Was it politics, then, or was it a woman? That is the question which confronted me. I was inclined from the first to the latter supposition. Political assassins are only too glad to do their work and fly. This murder had, on the contrary, been done most deliberately, and the perpetrator has left his tracks all over the room, showing he had been there all the time.''
\end{quote}
After stating the above, Sherlock Holmes concludes: \textit{``It was a woman''}.

Show the premises and logical inferences involved in deducing the conclusion.

  \begin{solution}\\
    P: \text{The murder was for robbery.} \\
    Q: \text{The murder was for politics.} \\
    R: \text{The murder was because of a woman.} \\
    S: \text{The murder was deliberate.} \\
    T: \text{The perpetrator has left his tracks.} \\
    U: \text{The murderer was a political assassin.} \\
    \\
    \text{The following premises are given in the quote:} \\
    1. $\neg P$ \\
    2. $(Q \land \neg R) \lor (\neg Q \land R)$ \\
    3. $U \rightarrow \neg (S \land T)$ \\
    4. $S \land T$ \\
    5. $Q \rightarrow U$ \\
    \\
    \begin{tabular}{|l|l|}
      \hline
      step & reason\\
      \hline
      6. $\neg U$ & Modus tollens on 3 and 4\\
      7. $\neg Q$ & Modus tollens on 5 and 6\\
      8. $\neg Q \land R$ & This proposition from 2 is True\\
      9. $R$ & Simplification on 8\\
      \hline
    \end{tabular}\\\\
    Therefore, Sherlock's deduction is correct.


  \end{solution}    
  
  \question Use rules of inference to show that if $\forall x(P(x) \implies (Q(x) \land S(x)))$ and $\forall x(P(x) \land R(x))$ are true, then $\forall x(R(x) \land S(x))$ is true.
  \begin{solution}
      % Enter your solution here.
      1. $\forall x(P(x) \implies (Q(x) \land S(x)))$\\
      2. $\forall x(P(x) \land R(x))$\\
      \begin{tabular}{|l|l|}
        \hline
        step & reason\\
        \hline
        3. $P(c) \implies (Q(c) \land S(c))$ & Universal instantiation on 1 \\
        4. $P(c) \land R(c)$ & Universal instantiation on 2\\
        5. $P(c)$ & Simplification on 4 \\
        6. $Q(c) \land S(c)$ & Modus ponens on 3 and 5\\
        7. $S(c)$ & Simplification on 6\\
        8. $R(c)$ & Simplicaficaion on 4\\
        9. $R(c) \land S(c)$ & Conjunction on 7 and 8\\
        10. $\forall x (R(x) \land S(x))$ & Universal generalization on 9 \\
        \hline
      \end{tabular}

    \end{solution}

\end{questions}
\end{document}
%%% Local Variables:
%%% mode: latex
%%% TeX-master: t
%%% End: